\chapter{INTRODU��O} 
\label{cap:ini}
\vspace{-2cm}

The Large Hadron Collider(LHC) had it's first operation in 2010 with proton-proton collisions with a 3,5 tera electronsvolts(TeV) center-of-mass energy. The LHC system has been updated since their first operation. The improvement aims collision improvement 

Atualmente com a evolu��o da tecnologia, ...


A tabela abaixo � uma tabela de exemplo de ...

\begin{table}[htbp]
\begin{minipage}[b]{0.61\linewidth}
\begin{spacing}{-0.5} %Espa�o entre o t�tulo e a tabela
\caption{Resultado para o sistema}
\label{tab:Resultado-Linear-DisjuntivoIEEE}
\end{spacing}
\end{minipage}
\begin{centering}
\begin{tabular}{c|c|c|c}
\cline{2-4}
\multicolumn{1}{c|}{} &   Est�gio 1   &   Est�gio 2   & Est�gio 3    \tabularnewline
\hline
\multirow{3}{*}{}     &  $n_{6-10}=1$ & $n_{20-23}=1$ & $n_{1-5}=1$\tabularnewline
%\cline{2-4}
N�mero de linhas      &   $n_{7-8}=2$ &               & $n_{3-24}=1$\tabularnewline
%\cline{2-4}
      $n_{ij}$        & $n_{10-12}=1$ &               &             \tabularnewline
%\cline{2-4}
\multirow{3}{*}{}     & $n_{11-13}=1$ &               &             \tabularnewline
\hline
%\hline
\multicolumn{4}{c}{Fun��o Objetivo $v$ = 220.2860}\tabularnewline
\hline
\multicolumn{4}{l}{\small{Fonte: Dados da pesquisa do autor.}}\tabularnewline
\end{tabular}
%\\
%{\small{Fonte: o pr�prio autor}}
\par\end{centering}
\end{table}

